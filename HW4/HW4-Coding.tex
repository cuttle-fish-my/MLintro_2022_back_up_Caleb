\documentclass[11pt]{article}

    \usepackage[breakable]{tcolorbox}
    \usepackage{parskip} % Stop auto-indenting (to mimic markdown behaviour)
    
    \usepackage{iftex}
    \ifPDFTeX
    	\usepackage[T1]{fontenc}
    	\usepackage{mathpazo}
    \else
    	\usepackage{fontspec}
    \fi

    % Basic figure setup, for now with no caption control since it's done
    % automatically by Pandoc (which extracts ![](path) syntax from Markdown).
    \usepackage{graphicx}
    % Maintain compatibility with old templates. Remove in nbconvert 6.0
    \let\Oldincludegraphics\includegraphics
    % Ensure that by default, figures have no caption (until we provide a
    % proper Figure object with a Caption API and a way to capture that
    % in the conversion process - todo).
    \usepackage{caption}
    \DeclareCaptionFormat{nocaption}{}
    \captionsetup{format=nocaption,aboveskip=0pt,belowskip=0pt}

    \usepackage[Export]{adjustbox} % Used to constrain images to a maximum size
    \adjustboxset{max size={0.9\linewidth}{0.9\paperheight}}
    \usepackage{float}
    \floatplacement{figure}{H} % forces figures to be placed at the correct location
    \usepackage{xcolor} % Allow colors to be defined
    \usepackage{enumerate} % Needed for markdown enumerations to work
    \usepackage{geometry} % Used to adjust the document margins
    \usepackage{amsmath} % Equations
    \usepackage{amssymb} % Equations
    \usepackage{textcomp} % defines textquotesingle
    % Hack from http://tex.stackexchange.com/a/47451/13684:
    \AtBeginDocument{%
        \def\PYZsq{\textquotesingle}% Upright quotes in Pygmentized code
    }
    \usepackage{upquote} % Upright quotes for verbatim code
    \usepackage{eurosym} % defines \euro
    \usepackage[mathletters]{ucs} % Extended unicode (utf-8) support
    \usepackage{fancyvrb} % verbatim replacement that allows latex
    \usepackage{grffile} % extends the file name processing of package graphics 
                         % to support a larger range
    \makeatletter % fix for grffile with XeLaTeX
    \def\Gread@@xetex#1{%
      \IfFileExists{"\Gin@base".bb}%
      {\Gread@eps{\Gin@base.bb}}%
      {\Gread@@xetex@aux#1}%
    }
    \makeatother

    % The hyperref package gives us a pdf with properly built
    % internal navigation ('pdf bookmarks' for the table of contents,
    % internal cross-reference links, web links for URLs, etc.)
    \usepackage{hyperref}
    % The default LaTeX title has an obnoxious amount of whitespace. By default,
    % titling removes some of it. It also provides customization options.
    \usepackage{titling}
    \usepackage{longtable} % longtable support required by pandoc >1.10
    \usepackage{booktabs}  % table support for pandoc > 1.12.2
    \usepackage[inline]{enumitem} % IRkernel/repr support (it uses the enumerate* environment)
    \usepackage[normalem]{ulem} % ulem is needed to support strikethroughs (\sout)
                                % normalem makes italics be italics, not underlines
    \usepackage{mathrsfs}
    

    
    % Colors for the hyperref package
    \definecolor{urlcolor}{rgb}{0,.145,.698}
    \definecolor{linkcolor}{rgb}{.71,0.21,0.01}
    \definecolor{citecolor}{rgb}{.12,.54,.11}

    % ANSI colors
    \definecolor{ansi-black}{HTML}{3E424D}
    \definecolor{ansi-black-intense}{HTML}{282C36}
    \definecolor{ansi-red}{HTML}{E75C58}
    \definecolor{ansi-red-intense}{HTML}{B22B31}
    \definecolor{ansi-green}{HTML}{00A250}
    \definecolor{ansi-green-intense}{HTML}{007427}
    \definecolor{ansi-yellow}{HTML}{DDB62B}
    \definecolor{ansi-yellow-intense}{HTML}{B27D12}
    \definecolor{ansi-blue}{HTML}{208FFB}
    \definecolor{ansi-blue-intense}{HTML}{0065CA}
    \definecolor{ansi-magenta}{HTML}{D160C4}
    \definecolor{ansi-magenta-intense}{HTML}{A03196}
    \definecolor{ansi-cyan}{HTML}{60C6C8}
    \definecolor{ansi-cyan-intense}{HTML}{258F8F}
    \definecolor{ansi-white}{HTML}{C5C1B4}
    \definecolor{ansi-white-intense}{HTML}{A1A6B2}
    \definecolor{ansi-default-inverse-fg}{HTML}{FFFFFF}
    \definecolor{ansi-default-inverse-bg}{HTML}{000000}

    % commands and environments needed by pandoc snippets
    % extracted from the output of `pandoc -s`
    \providecommand{\tightlist}{%
      \setlength{\itemsep}{0pt}\setlength{\parskip}{0pt}}
    \DefineVerbatimEnvironment{Highlighting}{Verbatim}{commandchars=\\\{\}}
    % Add ',fontsize=\small' for more characters per line
    \newenvironment{Shaded}{}{}
    \newcommand{\KeywordTok}[1]{\textcolor[rgb]{0.00,0.44,0.13}{\textbf{{#1}}}}
    \newcommand{\DataTypeTok}[1]{\textcolor[rgb]{0.56,0.13,0.00}{{#1}}}
    \newcommand{\DecValTok}[1]{\textcolor[rgb]{0.25,0.63,0.44}{{#1}}}
    \newcommand{\BaseNTok}[1]{\textcolor[rgb]{0.25,0.63,0.44}{{#1}}}
    \newcommand{\FloatTok}[1]{\textcolor[rgb]{0.25,0.63,0.44}{{#1}}}
    \newcommand{\CharTok}[1]{\textcolor[rgb]{0.25,0.44,0.63}{{#1}}}
    \newcommand{\StringTok}[1]{\textcolor[rgb]{0.25,0.44,0.63}{{#1}}}
    \newcommand{\CommentTok}[1]{\textcolor[rgb]{0.38,0.63,0.69}{\textit{{#1}}}}
    \newcommand{\OtherTok}[1]{\textcolor[rgb]{0.00,0.44,0.13}{{#1}}}
    \newcommand{\AlertTok}[1]{\textcolor[rgb]{1.00,0.00,0.00}{\textbf{{#1}}}}
    \newcommand{\FunctionTok}[1]{\textcolor[rgb]{0.02,0.16,0.49}{{#1}}}
    \newcommand{\RegionMarkerTok}[1]{{#1}}
    \newcommand{\ErrorTok}[1]{\textcolor[rgb]{1.00,0.00,0.00}{\textbf{{#1}}}}
    \newcommand{\NormalTok}[1]{{#1}}
    
    % Additional commands for more recent versions of Pandoc
    \newcommand{\ConstantTok}[1]{\textcolor[rgb]{0.53,0.00,0.00}{{#1}}}
    \newcommand{\SpecialCharTok}[1]{\textcolor[rgb]{0.25,0.44,0.63}{{#1}}}
    \newcommand{\VerbatimStringTok}[1]{\textcolor[rgb]{0.25,0.44,0.63}{{#1}}}
    \newcommand{\SpecialStringTok}[1]{\textcolor[rgb]{0.73,0.40,0.53}{{#1}}}
    \newcommand{\ImportTok}[1]{{#1}}
    \newcommand{\DocumentationTok}[1]{\textcolor[rgb]{0.73,0.13,0.13}{\textit{{#1}}}}
    \newcommand{\AnnotationTok}[1]{\textcolor[rgb]{0.38,0.63,0.69}{\textbf{\textit{{#1}}}}}
    \newcommand{\CommentVarTok}[1]{\textcolor[rgb]{0.38,0.63,0.69}{\textbf{\textit{{#1}}}}}
    \newcommand{\VariableTok}[1]{\textcolor[rgb]{0.10,0.09,0.49}{{#1}}}
    \newcommand{\ControlFlowTok}[1]{\textcolor[rgb]{0.00,0.44,0.13}{\textbf{{#1}}}}
    \newcommand{\OperatorTok}[1]{\textcolor[rgb]{0.40,0.40,0.40}{{#1}}}
    \newcommand{\BuiltInTok}[1]{{#1}}
    \newcommand{\ExtensionTok}[1]{{#1}}
    \newcommand{\PreprocessorTok}[1]{\textcolor[rgb]{0.74,0.48,0.00}{{#1}}}
    \newcommand{\AttributeTok}[1]{\textcolor[rgb]{0.49,0.56,0.16}{{#1}}}
    \newcommand{\InformationTok}[1]{\textcolor[rgb]{0.38,0.63,0.69}{\textbf{\textit{{#1}}}}}
    \newcommand{\WarningTok}[1]{\textcolor[rgb]{0.38,0.63,0.69}{\textbf{\textit{{#1}}}}}
    
    
    % Define a nice break command that doesn't care if a line doesn't already
    % exist.
    \def\br{\hspace*{\fill} \\* }
    % Math Jax compatibility definitions
    \def\gt{>}
    \def\lt{<}
    \let\Oldtex\TeX
    \let\Oldlatex\LaTeX
    \renewcommand{\TeX}{\textrm{\Oldtex}}
    \renewcommand{\LaTeX}{\textrm{\Oldlatex}}
    % Document parameters
    % Document title
    \title{HW4-Coding}
    
    
    
    
    
% Pygments definitions
\makeatletter
\def\PY@reset{\let\PY@it=\relax \let\PY@bf=\relax%
    \let\PY@ul=\relax \let\PY@tc=\relax%
    \let\PY@bc=\relax \let\PY@ff=\relax}
\def\PY@tok#1{\csname PY@tok@#1\endcsname}
\def\PY@toks#1+{\ifx\relax#1\empty\else%
    \PY@tok{#1}\expandafter\PY@toks\fi}
\def\PY@do#1{\PY@bc{\PY@tc{\PY@ul{%
    \PY@it{\PY@bf{\PY@ff{#1}}}}}}}
\def\PY#1#2{\PY@reset\PY@toks#1+\relax+\PY@do{#2}}

\expandafter\def\csname PY@tok@gd\endcsname{\def\PY@tc##1{\textcolor[rgb]{0.63,0.00,0.00}{##1}}}
\expandafter\def\csname PY@tok@gu\endcsname{\let\PY@bf=\textbf\def\PY@tc##1{\textcolor[rgb]{0.50,0.00,0.50}{##1}}}
\expandafter\def\csname PY@tok@gt\endcsname{\def\PY@tc##1{\textcolor[rgb]{0.00,0.27,0.87}{##1}}}
\expandafter\def\csname PY@tok@gs\endcsname{\let\PY@bf=\textbf}
\expandafter\def\csname PY@tok@gr\endcsname{\def\PY@tc##1{\textcolor[rgb]{1.00,0.00,0.00}{##1}}}
\expandafter\def\csname PY@tok@cm\endcsname{\let\PY@it=\textit\def\PY@tc##1{\textcolor[rgb]{0.25,0.50,0.50}{##1}}}
\expandafter\def\csname PY@tok@vg\endcsname{\def\PY@tc##1{\textcolor[rgb]{0.10,0.09,0.49}{##1}}}
\expandafter\def\csname PY@tok@vi\endcsname{\def\PY@tc##1{\textcolor[rgb]{0.10,0.09,0.49}{##1}}}
\expandafter\def\csname PY@tok@vm\endcsname{\def\PY@tc##1{\textcolor[rgb]{0.10,0.09,0.49}{##1}}}
\expandafter\def\csname PY@tok@mh\endcsname{\def\PY@tc##1{\textcolor[rgb]{0.40,0.40,0.40}{##1}}}
\expandafter\def\csname PY@tok@cs\endcsname{\let\PY@it=\textit\def\PY@tc##1{\textcolor[rgb]{0.25,0.50,0.50}{##1}}}
\expandafter\def\csname PY@tok@ge\endcsname{\let\PY@it=\textit}
\expandafter\def\csname PY@tok@vc\endcsname{\def\PY@tc##1{\textcolor[rgb]{0.10,0.09,0.49}{##1}}}
\expandafter\def\csname PY@tok@il\endcsname{\def\PY@tc##1{\textcolor[rgb]{0.40,0.40,0.40}{##1}}}
\expandafter\def\csname PY@tok@go\endcsname{\def\PY@tc##1{\textcolor[rgb]{0.53,0.53,0.53}{##1}}}
\expandafter\def\csname PY@tok@cp\endcsname{\def\PY@tc##1{\textcolor[rgb]{0.74,0.48,0.00}{##1}}}
\expandafter\def\csname PY@tok@gi\endcsname{\def\PY@tc##1{\textcolor[rgb]{0.00,0.63,0.00}{##1}}}
\expandafter\def\csname PY@tok@gh\endcsname{\let\PY@bf=\textbf\def\PY@tc##1{\textcolor[rgb]{0.00,0.00,0.50}{##1}}}
\expandafter\def\csname PY@tok@ni\endcsname{\let\PY@bf=\textbf\def\PY@tc##1{\textcolor[rgb]{0.60,0.60,0.60}{##1}}}
\expandafter\def\csname PY@tok@nl\endcsname{\def\PY@tc##1{\textcolor[rgb]{0.63,0.63,0.00}{##1}}}
\expandafter\def\csname PY@tok@nn\endcsname{\let\PY@bf=\textbf\def\PY@tc##1{\textcolor[rgb]{0.00,0.00,1.00}{##1}}}
\expandafter\def\csname PY@tok@no\endcsname{\def\PY@tc##1{\textcolor[rgb]{0.53,0.00,0.00}{##1}}}
\expandafter\def\csname PY@tok@na\endcsname{\def\PY@tc##1{\textcolor[rgb]{0.49,0.56,0.16}{##1}}}
\expandafter\def\csname PY@tok@nb\endcsname{\def\PY@tc##1{\textcolor[rgb]{0.00,0.50,0.00}{##1}}}
\expandafter\def\csname PY@tok@nc\endcsname{\let\PY@bf=\textbf\def\PY@tc##1{\textcolor[rgb]{0.00,0.00,1.00}{##1}}}
\expandafter\def\csname PY@tok@nd\endcsname{\def\PY@tc##1{\textcolor[rgb]{0.67,0.13,1.00}{##1}}}
\expandafter\def\csname PY@tok@ne\endcsname{\let\PY@bf=\textbf\def\PY@tc##1{\textcolor[rgb]{0.82,0.25,0.23}{##1}}}
\expandafter\def\csname PY@tok@nf\endcsname{\def\PY@tc##1{\textcolor[rgb]{0.00,0.00,1.00}{##1}}}
\expandafter\def\csname PY@tok@si\endcsname{\let\PY@bf=\textbf\def\PY@tc##1{\textcolor[rgb]{0.73,0.40,0.53}{##1}}}
\expandafter\def\csname PY@tok@s2\endcsname{\def\PY@tc##1{\textcolor[rgb]{0.73,0.13,0.13}{##1}}}
\expandafter\def\csname PY@tok@nt\endcsname{\let\PY@bf=\textbf\def\PY@tc##1{\textcolor[rgb]{0.00,0.50,0.00}{##1}}}
\expandafter\def\csname PY@tok@nv\endcsname{\def\PY@tc##1{\textcolor[rgb]{0.10,0.09,0.49}{##1}}}
\expandafter\def\csname PY@tok@s1\endcsname{\def\PY@tc##1{\textcolor[rgb]{0.73,0.13,0.13}{##1}}}
\expandafter\def\csname PY@tok@dl\endcsname{\def\PY@tc##1{\textcolor[rgb]{0.73,0.13,0.13}{##1}}}
\expandafter\def\csname PY@tok@ch\endcsname{\let\PY@it=\textit\def\PY@tc##1{\textcolor[rgb]{0.25,0.50,0.50}{##1}}}
\expandafter\def\csname PY@tok@m\endcsname{\def\PY@tc##1{\textcolor[rgb]{0.40,0.40,0.40}{##1}}}
\expandafter\def\csname PY@tok@gp\endcsname{\let\PY@bf=\textbf\def\PY@tc##1{\textcolor[rgb]{0.00,0.00,0.50}{##1}}}
\expandafter\def\csname PY@tok@sh\endcsname{\def\PY@tc##1{\textcolor[rgb]{0.73,0.13,0.13}{##1}}}
\expandafter\def\csname PY@tok@ow\endcsname{\let\PY@bf=\textbf\def\PY@tc##1{\textcolor[rgb]{0.67,0.13,1.00}{##1}}}
\expandafter\def\csname PY@tok@sx\endcsname{\def\PY@tc##1{\textcolor[rgb]{0.00,0.50,0.00}{##1}}}
\expandafter\def\csname PY@tok@bp\endcsname{\def\PY@tc##1{\textcolor[rgb]{0.00,0.50,0.00}{##1}}}
\expandafter\def\csname PY@tok@c1\endcsname{\let\PY@it=\textit\def\PY@tc##1{\textcolor[rgb]{0.25,0.50,0.50}{##1}}}
\expandafter\def\csname PY@tok@fm\endcsname{\def\PY@tc##1{\textcolor[rgb]{0.00,0.00,1.00}{##1}}}
\expandafter\def\csname PY@tok@o\endcsname{\def\PY@tc##1{\textcolor[rgb]{0.40,0.40,0.40}{##1}}}
\expandafter\def\csname PY@tok@kc\endcsname{\let\PY@bf=\textbf\def\PY@tc##1{\textcolor[rgb]{0.00,0.50,0.00}{##1}}}
\expandafter\def\csname PY@tok@c\endcsname{\let\PY@it=\textit\def\PY@tc##1{\textcolor[rgb]{0.25,0.50,0.50}{##1}}}
\expandafter\def\csname PY@tok@mf\endcsname{\def\PY@tc##1{\textcolor[rgb]{0.40,0.40,0.40}{##1}}}
\expandafter\def\csname PY@tok@err\endcsname{\def\PY@bc##1{\setlength{\fboxsep}{0pt}\fcolorbox[rgb]{1.00,0.00,0.00}{1,1,1}{\strut ##1}}}
\expandafter\def\csname PY@tok@mb\endcsname{\def\PY@tc##1{\textcolor[rgb]{0.40,0.40,0.40}{##1}}}
\expandafter\def\csname PY@tok@ss\endcsname{\def\PY@tc##1{\textcolor[rgb]{0.10,0.09,0.49}{##1}}}
\expandafter\def\csname PY@tok@sr\endcsname{\def\PY@tc##1{\textcolor[rgb]{0.73,0.40,0.53}{##1}}}
\expandafter\def\csname PY@tok@mo\endcsname{\def\PY@tc##1{\textcolor[rgb]{0.40,0.40,0.40}{##1}}}
\expandafter\def\csname PY@tok@kd\endcsname{\let\PY@bf=\textbf\def\PY@tc##1{\textcolor[rgb]{0.00,0.50,0.00}{##1}}}
\expandafter\def\csname PY@tok@mi\endcsname{\def\PY@tc##1{\textcolor[rgb]{0.40,0.40,0.40}{##1}}}
\expandafter\def\csname PY@tok@kn\endcsname{\let\PY@bf=\textbf\def\PY@tc##1{\textcolor[rgb]{0.00,0.50,0.00}{##1}}}
\expandafter\def\csname PY@tok@cpf\endcsname{\let\PY@it=\textit\def\PY@tc##1{\textcolor[rgb]{0.25,0.50,0.50}{##1}}}
\expandafter\def\csname PY@tok@kr\endcsname{\let\PY@bf=\textbf\def\PY@tc##1{\textcolor[rgb]{0.00,0.50,0.00}{##1}}}
\expandafter\def\csname PY@tok@s\endcsname{\def\PY@tc##1{\textcolor[rgb]{0.73,0.13,0.13}{##1}}}
\expandafter\def\csname PY@tok@kp\endcsname{\def\PY@tc##1{\textcolor[rgb]{0.00,0.50,0.00}{##1}}}
\expandafter\def\csname PY@tok@w\endcsname{\def\PY@tc##1{\textcolor[rgb]{0.73,0.73,0.73}{##1}}}
\expandafter\def\csname PY@tok@kt\endcsname{\def\PY@tc##1{\textcolor[rgb]{0.69,0.00,0.25}{##1}}}
\expandafter\def\csname PY@tok@sc\endcsname{\def\PY@tc##1{\textcolor[rgb]{0.73,0.13,0.13}{##1}}}
\expandafter\def\csname PY@tok@sb\endcsname{\def\PY@tc##1{\textcolor[rgb]{0.73,0.13,0.13}{##1}}}
\expandafter\def\csname PY@tok@sa\endcsname{\def\PY@tc##1{\textcolor[rgb]{0.73,0.13,0.13}{##1}}}
\expandafter\def\csname PY@tok@k\endcsname{\let\PY@bf=\textbf\def\PY@tc##1{\textcolor[rgb]{0.00,0.50,0.00}{##1}}}
\expandafter\def\csname PY@tok@se\endcsname{\let\PY@bf=\textbf\def\PY@tc##1{\textcolor[rgb]{0.73,0.40,0.13}{##1}}}
\expandafter\def\csname PY@tok@sd\endcsname{\let\PY@it=\textit\def\PY@tc##1{\textcolor[rgb]{0.73,0.13,0.13}{##1}}}

\def\PYZbs{\char`\\}
\def\PYZus{\char`\_}
\def\PYZob{\char`\{}
\def\PYZcb{\char`\}}
\def\PYZca{\char`\^}
\def\PYZam{\char`\&}
\def\PYZlt{\char`\<}
\def\PYZgt{\char`\>}
\def\PYZsh{\char`\#}
\def\PYZpc{\char`\%}
\def\PYZdl{\char`\$}
\def\PYZhy{\char`\-}
\def\PYZsq{\char`\'}
\def\PYZdq{\char`\"}
\def\PYZti{\char`\~}
% for compatibility with earlier versions
\def\PYZat{@}
\def\PYZlb{[}
\def\PYZrb{]}
\makeatother


    % For linebreaks inside Verbatim environment from package fancyvrb. 
    \makeatletter
        \newbox\Wrappedcontinuationbox 
        \newbox\Wrappedvisiblespacebox 
        \newcommand*\Wrappedvisiblespace {\textcolor{red}{\textvisiblespace}} 
        \newcommand*\Wrappedcontinuationsymbol {\textcolor{red}{\llap{\tiny$\m@th\hookrightarrow$}}} 
        \newcommand*\Wrappedcontinuationindent {3ex } 
        \newcommand*\Wrappedafterbreak {\kern\Wrappedcontinuationindent\copy\Wrappedcontinuationbox} 
        % Take advantage of the already applied Pygments mark-up to insert 
        % potential linebreaks for TeX processing. 
        %        {, <, #, %, $, ' and ": go to next line. 
        %        _, }, ^, &, >, - and ~: stay at end of broken line. 
        % Use of \textquotesingle for straight quote. 
        \newcommand*\Wrappedbreaksatspecials {% 
            \def\PYGZus{\discretionary{\char`\_}{\Wrappedafterbreak}{\char`\_}}% 
            \def\PYGZob{\discretionary{}{\Wrappedafterbreak\char`\{}{\char`\{}}% 
            \def\PYGZcb{\discretionary{\char`\}}{\Wrappedafterbreak}{\char`\}}}% 
            \def\PYGZca{\discretionary{\char`\^}{\Wrappedafterbreak}{\char`\^}}% 
            \def\PYGZam{\discretionary{\char`\&}{\Wrappedafterbreak}{\char`\&}}% 
            \def\PYGZlt{\discretionary{}{\Wrappedafterbreak\char`\<}{\char`\<}}% 
            \def\PYGZgt{\discretionary{\char`\>}{\Wrappedafterbreak}{\char`\>}}% 
            \def\PYGZsh{\discretionary{}{\Wrappedafterbreak\char`\#}{\char`\#}}% 
            \def\PYGZpc{\discretionary{}{\Wrappedafterbreak\char`\%}{\char`\%}}% 
            \def\PYGZdl{\discretionary{}{\Wrappedafterbreak\char`\$}{\char`\$}}% 
            \def\PYGZhy{\discretionary{\char`\-}{\Wrappedafterbreak}{\char`\-}}% 
            \def\PYGZsq{\discretionary{}{\Wrappedafterbreak\textquotesingle}{\textquotesingle}}% 
            \def\PYGZdq{\discretionary{}{\Wrappedafterbreak\char`\"}{\char`\"}}% 
            \def\PYGZti{\discretionary{\char`\~}{\Wrappedafterbreak}{\char`\~}}% 
        } 
        % Some characters . , ; ? ! / are not pygmentized. 
        % This macro makes them "active" and they will insert potential linebreaks 
        \newcommand*\Wrappedbreaksatpunct {% 
            \lccode`\~`\.\lowercase{\def~}{\discretionary{\hbox{\char`\.}}{\Wrappedafterbreak}{\hbox{\char`\.}}}% 
            \lccode`\~`\,\lowercase{\def~}{\discretionary{\hbox{\char`\,}}{\Wrappedafterbreak}{\hbox{\char`\,}}}% 
            \lccode`\~`\;\lowercase{\def~}{\discretionary{\hbox{\char`\;}}{\Wrappedafterbreak}{\hbox{\char`\;}}}% 
            \lccode`\~`\:\lowercase{\def~}{\discretionary{\hbox{\char`\:}}{\Wrappedafterbreak}{\hbox{\char`\:}}}% 
            \lccode`\~`\?\lowercase{\def~}{\discretionary{\hbox{\char`\?}}{\Wrappedafterbreak}{\hbox{\char`\?}}}% 
            \lccode`\~`\!\lowercase{\def~}{\discretionary{\hbox{\char`\!}}{\Wrappedafterbreak}{\hbox{\char`\!}}}% 
            \lccode`\~`\/\lowercase{\def~}{\discretionary{\hbox{\char`\/}}{\Wrappedafterbreak}{\hbox{\char`\/}}}% 
            \catcode`\.\active
            \catcode`\,\active 
            \catcode`\;\active
            \catcode`\:\active
            \catcode`\?\active
            \catcode`\!\active
            \catcode`\/\active 
            \lccode`\~`\~ 	
        }
    \makeatother

    \let\OriginalVerbatim=\Verbatim
    \makeatletter
    \renewcommand{\Verbatim}[1][1]{%
        %\parskip\z@skip
        \sbox\Wrappedcontinuationbox {\Wrappedcontinuationsymbol}%
        \sbox\Wrappedvisiblespacebox {\FV@SetupFont\Wrappedvisiblespace}%
        \def\FancyVerbFormatLine ##1{\hsize\linewidth
            \vtop{\raggedright\hyphenpenalty\z@\exhyphenpenalty\z@
                \doublehyphendemerits\z@\finalhyphendemerits\z@
                \strut ##1\strut}%
        }%
        % If the linebreak is at a space, the latter will be displayed as visible
        % space at end of first line, and a continuation symbol starts next line.
        % Stretch/shrink are however usually zero for typewriter font.
        \def\FV@Space {%
            \nobreak\hskip\z@ plus\fontdimen3\font minus\fontdimen4\font
            \discretionary{\copy\Wrappedvisiblespacebox}{\Wrappedafterbreak}
            {\kern\fontdimen2\font}%
        }%
        
        % Allow breaks at special characters using \PYG... macros.
        \Wrappedbreaksatspecials
        % Breaks at punctuation characters . , ; ? ! and / need catcode=\active 	
        \OriginalVerbatim[#1,codes*=\Wrappedbreaksatpunct]%
    }
    \makeatother

    % Exact colors from NB
    \definecolor{incolor}{HTML}{303F9F}
    \definecolor{outcolor}{HTML}{D84315}
    \definecolor{cellborder}{HTML}{CFCFCF}
    \definecolor{cellbackground}{HTML}{F7F7F7}
    
    % prompt
    \makeatletter
    \newcommand{\boxspacing}{\kern\kvtcb@left@rule\kern\kvtcb@boxsep}
    \makeatother
    \newcommand{\prompt}[4]{
        \ttfamily\llap{{\color{#2}[#3]:\hspace{3pt}#4}}\vspace{-\baselineskip}
    }
    

    
    % Prevent overflowing lines due to hard-to-break entities
    \sloppy 
    % Setup hyperref package
    \hypersetup{
      breaklinks=true,  % so long urls are correctly broken across lines
      colorlinks=true,
      urlcolor=urlcolor,
      linkcolor=linkcolor,
      citecolor=citecolor,
      }
    % Slightly bigger margins than the latex defaults
    
    \geometry{verbose,tmargin=1in,bmargin=1in,lmargin=1in,rmargin=1in}
    
    

\begin{document}
    
    \maketitle
    
    

    
    \hypertarget{fit-gmm-by-em-algorithm}{%
\section{Fit GMM by EM algorithm}\label{fit-gmm-by-em-algorithm}}

\emph{Please convert your coding notebook from .ipynb into {.pdf} and
concatenate it with your writing part.}

    You need to implement the EM algorithm, using closed-forms derived in
`./HW4-Writing.pdf/2(b)'.

    \begin{tcolorbox}[breakable, size=fbox, boxrule=1pt, pad at break*=1mm,colback=cellbackground, colframe=cellborder]
\prompt{In}{incolor}{63}{\boxspacing}
\begin{Verbatim}[commandchars=\\\{\}]
\PY{k+kn}{import} \PY{n+nn}{matplotlib}\PY{n+nn}{.}\PY{n+nn}{pyplot} \PY{k}{as} \PY{n+nn}{plt}
\PY{c+c1}{\PYZsh{} Do NOT change this cell.}
\PY{k+kn}{import} \PY{n+nn}{numpy} \PY{k}{as} \PY{n+nn}{np}
\end{Verbatim}
\end{tcolorbox}

    \begin{tcolorbox}[breakable, size=fbox, boxrule=1pt, pad at break*=1mm,colback=cellbackground, colframe=cellborder]
\prompt{In}{incolor}{64}{\boxspacing}
\begin{Verbatim}[commandchars=\\\{\}]
\PY{c+c1}{\PYZsh{} E\PYZhy{}step: compute posterior probabilities h\PYZus{}\PYZob{}li\PYZcb{}}
\PY{k}{def} \PY{n+nf}{probabilities}\PY{p}{(}\PY{n}{data}\PY{p}{,} \PY{n}{weights}\PY{p}{,} \PY{n}{means}\PY{p}{,} \PY{n}{covariances}\PY{p}{)}\PY{p}{:}
    \PY{l+s+sd}{\PYZsq{}\PYZsq{}\PYZsq{}}
\PY{l+s+sd}{    :param data: size: (N, D)}
\PY{l+s+sd}{    :param weights: corresponds to \PYZbs{}pi, size: (K,)}
\PY{l+s+sd}{    :param means: corresponds to \PYZbs{}mu, size: (K, D)}
\PY{l+s+sd}{    :param covariances: corresponds to \PYZbs{}sigma, size: (K, D, D)}
\PY{l+s+sd}{    :return prob\PYZus{}matrix: a matrix filled by posterior probability h\PYZus{}\PYZob{}ik\PYZcb{}, size (N, K)}
\PY{l+s+sd}{    \PYZsq{}\PYZsq{}\PYZsq{}}
    \PY{n}{num\PYZus{}data} \PY{o}{=} \PY{n+nb}{len}\PY{p}{(}\PY{n}{data}\PY{p}{)}
    \PY{n}{num\PYZus{}clusters} \PY{o}{=} \PY{n+nb}{len}\PY{p}{(}\PY{n}{means}\PY{p}{)}
    \PY{n}{prob\PYZus{}matrix} \PY{o}{=} \PY{n}{np}\PY{o}{.}\PY{n}{zeros}\PY{p}{(}\PY{p}{(}\PY{n}{num\PYZus{}data}\PY{p}{,} \PY{n}{num\PYZus{}clusters}\PY{p}{)}\PY{p}{)}
    \PY{c+c1}{\PYZsh{}\PYZsh{}\PYZsh{}\PYZsh{}\PYZsh{}\PYZsh{}\PYZsh{}\PYZsh{}\PYZsh{}\PYZsh{} START YOUR CODE HERE \PYZsh{}\PYZsh{}\PYZsh{}\PYZsh{}\PYZsh{}\PYZsh{}\PYZsh{}\PYZsh{}\PYZsh{}\PYZsh{}}
    \PY{n}{cov\PYZus{}inv} \PY{o}{=} \PY{n}{np}\PY{o}{.}\PY{n}{zeros\PYZus{}like}\PY{p}{(}\PY{n}{covariances}\PY{p}{)}
    \PY{n}{cov\PYZus{}det} \PY{o}{=} \PY{n}{np}\PY{o}{.}\PY{n}{zeros}\PY{p}{(}\PY{n}{num\PYZus{}clusters}\PY{p}{)}
    \PY{k}{for} \PY{n}{k} \PY{o+ow}{in} \PY{n+nb}{range}\PY{p}{(}\PY{n}{num\PYZus{}clusters}\PY{p}{)}\PY{p}{:}
        \PY{n}{cov\PYZus{}inv}\PY{p}{[}\PY{n}{k}\PY{p}{]} \PY{o}{=} \PY{n}{np}\PY{o}{.}\PY{n}{linalg}\PY{o}{.}\PY{n}{inv}\PY{p}{(}\PY{n}{covariances}\PY{p}{[}\PY{n}{k}\PY{p}{]}\PY{p}{)}
        \PY{n}{cov\PYZus{}det}\PY{p}{[}\PY{n}{k}\PY{p}{]} \PY{o}{=} \PY{n}{np}\PY{o}{.}\PY{n}{linalg}\PY{o}{.}\PY{n}{det}\PY{p}{(}\PY{n}{covariances}\PY{p}{[}\PY{n}{k}\PY{p}{]}\PY{p}{)}
    \PY{n}{cov\PYZus{}det} \PY{o}{=} \PY{n}{np}\PY{o}{.}\PY{n}{power}\PY{p}{(}\PY{n}{cov\PYZus{}det}\PY{p}{,}\PY{p}{[}\PY{o}{\PYZhy{}}\PY{l+m+mi}{1}\PY{o}{/}\PY{l+m+mi}{2}\PY{p}{]}\PY{p}{)}
    \PY{k}{for} \PY{n}{l} \PY{o+ow}{in} \PY{n+nb}{range}\PY{p}{(}\PY{n}{num\PYZus{}data}\PY{p}{)}\PY{p}{:}
        \PY{n}{exp\PYZus{}value} \PY{o}{=} \PY{n}{np}\PY{o}{.}\PY{n}{zeros}\PY{p}{(}\PY{n}{num\PYZus{}clusters}\PY{p}{)}
        \PY{k}{for} \PY{n}{i} \PY{o+ow}{in} \PY{n+nb}{range}\PY{p}{(}\PY{n}{num\PYZus{}clusters}\PY{p}{)}\PY{p}{:}
            \PY{n}{DM} \PY{o}{=} \PY{n}{data}\PY{p}{[}\PY{n}{l}\PY{p}{]} \PY{o}{\PYZhy{}} \PY{n}{means}\PY{p}{[}\PY{n}{i}\PY{p}{]}
            \PY{n}{exp\PYZus{}value}\PY{p}{[}\PY{n}{i}\PY{p}{]} \PY{o}{=} \PY{o}{\PYZhy{}}\PY{l+m+mi}{1} \PY{o}{/} \PY{l+m+mi}{2} \PY{o}{*} \PY{p}{(}\PY{n}{np}\PY{o}{.}\PY{n}{matrix}\PY{o}{.}\PY{n}{transpose}\PY{p}{(}\PY{n}{DM}\PY{p}{)} \PY{o}{@} \PY{n}{cov\PYZus{}inv}\PY{p}{[}\PY{n}{i}\PY{p}{]} \PY{o}{@} \PY{n}{DM}\PY{p}{)}
        \PY{n}{exp\PYZus{}value} \PY{o}{=} \PY{n}{np}\PY{o}{.}\PY{n}{exp}\PY{p}{(}\PY{n}{exp\PYZus{}value}\PY{p}{)}
        \PY{n}{denominator} \PY{o}{=} \PY{n}{np}\PY{o}{.}\PY{n}{matrix}\PY{o}{.}\PY{n}{transpose}\PY{p}{(}\PY{n}{cov\PYZus{}det} \PY{o}{*} \PY{n}{exp\PYZus{}value}\PY{p}{)} \PY{o}{@} \PY{n}{weights}
        \PY{n}{numerator} \PY{o}{=} \PY{n}{cov\PYZus{}det} \PY{o}{*} \PY{n}{exp\PYZus{}value} \PY{o}{*} \PY{n}{weights}
        \PY{n}{prob\PYZus{}matrix}\PY{p}{[}\PY{n}{l}\PY{p}{]} \PY{o}{=} \PY{n}{numerator} \PY{o}{/} \PY{n}{denominator}
    \PY{c+c1}{\PYZsh{}\PYZsh{}\PYZsh{}\PYZsh{}\PYZsh{}\PYZsh{}\PYZsh{}\PYZsh{}\PYZsh{}\PYZsh{} END YOUR CODE HERE \PYZsh{}\PYZsh{}\PYZsh{}\PYZsh{}\PYZsh{}\PYZsh{}\PYZsh{}\PYZsh{}\PYZsh{}\PYZsh{}}
    \PY{k}{return} \PY{n}{prob\PYZus{}matrix}
\end{Verbatim}
\end{tcolorbox}

    \begin{tcolorbox}[breakable, size=fbox, boxrule=1pt, pad at break*=1mm,colback=cellbackground, colframe=cellborder]
\prompt{In}{incolor}{65}{\boxspacing}
\begin{Verbatim}[commandchars=\\\{\}]
\PY{c+c1}{\PYZsh{} M\PYZhy{}step: update weights \PYZbs{}pi, means \PYZbs{}mu, and covariances \PYZbs{}sigma}
\PY{k}{def} \PY{n+nf}{updates}\PY{p}{(}\PY{n}{data}\PY{p}{,} \PY{n}{prob\PYZus{}matrix}\PY{p}{,} \PY{n}{weights}\PY{p}{,} \PY{n}{means}\PY{p}{,} \PY{n}{covariances}\PY{p}{)}\PY{p}{:}
    \PY{n}{num\PYZus{}clusters} \PY{o}{=} \PY{n+nb}{len}\PY{p}{(}\PY{n}{means}\PY{p}{)}
    \PY{n}{num\PYZus{}data} \PY{o}{=} \PY{n+nb}{len}\PY{p}{(}\PY{n}{data}\PY{p}{)}
    \PY{n}{dim} \PY{o}{=} \PY{n}{data}\PY{o}{.}\PY{n}{shape}\PY{p}{[}\PY{l+m+mi}{1}\PY{p}{]}

    \PY{c+c1}{\PYZsh{}\PYZsh{}\PYZsh{}\PYZsh{}\PYZsh{}\PYZsh{}\PYZsh{}\PYZsh{}\PYZsh{}\PYZsh{} START YOUR CODE HERE \PYZsh{}\PYZsh{}\PYZsh{}\PYZsh{}\PYZsh{}\PYZsh{}\PYZsh{}\PYZsh{}\PYZsh{}\PYZsh{}}
    \PY{n}{prob\PYZus{}sum} \PY{o}{=} \PY{n}{np}\PY{o}{.}\PY{n}{sum}\PY{p}{(}\PY{n}{prob\PYZus{}matrix}\PY{p}{,} \PY{n}{axis}\PY{o}{=}\PY{l+m+mi}{0}\PY{p}{)}
    \PY{n}{weights} \PY{o}{=} \PY{n}{prob\PYZus{}sum} \PY{o}{/} \PY{n}{num\PYZus{}data}
    \PY{k}{for} \PY{n}{i} \PY{o+ow}{in} \PY{n+nb}{range}\PY{p}{(}\PY{n}{num\PYZus{}clusters}\PY{p}{)}\PY{p}{:}
        \PY{n}{means}\PY{p}{[}\PY{n}{i}\PY{p}{]} \PY{o}{=} \PY{n}{np}\PY{o}{.}\PY{n}{matrix}\PY{o}{.}\PY{n}{transpose}\PY{p}{(}\PY{n}{prob\PYZus{}matrix}\PY{p}{[}\PY{o}{.}\PY{o}{.}\PY{o}{.}\PY{p}{,} \PY{n}{i}\PY{p}{]}\PY{p}{)} \PY{o}{@} \PY{n}{data} \PY{o}{/} \PY{n}{prob\PYZus{}sum}\PY{p}{[}\PY{n}{i}\PY{p}{]}
        \PY{n}{numerator} \PY{o}{=} \PY{n}{np}\PY{o}{.}\PY{n}{zeros}\PY{p}{(}\PY{p}{[}\PY{n}{dim}\PY{p}{,} \PY{n}{dim}\PY{p}{]}\PY{p}{)}
        \PY{k}{for} \PY{n}{l} \PY{o+ow}{in} \PY{n+nb}{range}\PY{p}{(}\PY{n}{num\PYZus{}data}\PY{p}{)}\PY{p}{:}
            \PY{n}{DM} \PY{o}{=} \PY{n}{np}\PY{o}{.}\PY{n}{expand\PYZus{}dims}\PY{p}{(}\PY{n}{data}\PY{p}{[}\PY{n}{l}\PY{p}{]} \PY{o}{\PYZhy{}} \PY{n}{means}\PY{p}{[}\PY{n}{i}\PY{p}{]}\PY{p}{,} \PY{n}{axis}\PY{o}{=}\PY{l+m+mi}{1}\PY{p}{)}
            \PY{n}{numerator} \PY{o}{+}\PY{o}{=} \PY{n}{prob\PYZus{}matrix}\PY{p}{[}\PY{n}{l}\PY{p}{,} \PY{n}{i}\PY{p}{]} \PY{o}{*} \PY{p}{(}\PY{n}{DM} \PY{o}{@} \PY{n}{np}\PY{o}{.}\PY{n}{matrix}\PY{o}{.}\PY{n}{transpose}\PY{p}{(}\PY{n}{DM}\PY{p}{)}\PY{p}{)}
        \PY{n}{covariances}\PY{p}{[}\PY{n}{i}\PY{p}{]} \PY{o}{=} \PY{n}{numerator} \PY{o}{/} \PY{n}{prob\PYZus{}sum}\PY{p}{[}\PY{n}{i}\PY{p}{]}
    \PY{c+c1}{\PYZsh{}\PYZsh{}\PYZsh{}\PYZsh{}\PYZsh{}\PYZsh{}\PYZsh{}\PYZsh{}\PYZsh{}\PYZsh{} END YOUR CODE HERE \PYZsh{}\PYZsh{}\PYZsh{}\PYZsh{}\PYZsh{}\PYZsh{}\PYZsh{}\PYZsh{}\PYZsh{}\PYZsh{}}

    \PY{k}{return} \PY{n}{weights}\PY{p}{,} \PY{n}{means}\PY{p}{,} \PY{n}{covariances}
\end{Verbatim}
\end{tcolorbox}

    To help us develop and test our implementation, we will generate some
observations from a mixture of Gaussians and then run our EM algorithm
to discover the mixture components. We'll begin with a function to
generate the data, and a quick plot to visualize its output for a
2-dimensional mixture of three Gaussians.

    \begin{tcolorbox}[breakable, size=fbox, boxrule=1pt, pad at break*=1mm,colback=cellbackground, colframe=cellborder]
\prompt{In}{incolor}{66}{\boxspacing}
\begin{Verbatim}[commandchars=\\\{\}]
\PY{c+c1}{\PYZsh{} Generate dataset}

\PY{k}{def} \PY{n+nf}{generate\PYZus{}MoG\PYZus{}data}\PY{p}{(}\PY{n}{num\PYZus{}data}\PY{p}{,} \PY{n}{means}\PY{p}{,} \PY{n}{covariances}\PY{p}{,} \PY{n}{weights}\PY{p}{)}\PY{p}{:}
    \PY{l+s+sd}{\PYZdq{}\PYZdq{}\PYZdq{} Creates a list of data points \PYZdq{}\PYZdq{}\PYZdq{}}
    \PY{n}{num\PYZus{}clusters} \PY{o}{=} \PY{n+nb}{len}\PY{p}{(}\PY{n}{weights}\PY{p}{)}
    \PY{n}{data} \PY{o}{=} \PY{p}{[}\PY{p}{]}
    \PY{k}{for} \PY{n}{i} \PY{o+ow}{in} \PY{n+nb}{range}\PY{p}{(}\PY{n}{num\PYZus{}data}\PY{p}{)}\PY{p}{:}
        \PY{c+c1}{\PYZsh{}  Use np.random.choice and weights to pick a cluster id greater than}
        \PY{c+c1}{\PYZsh{}  or equal to 0 and less than num\PYZus{}clusters.}
        \PY{n}{k} \PY{o}{=} \PY{n}{np}\PY{o}{.}\PY{n}{random}\PY{o}{.}\PY{n}{choice}\PY{p}{(}\PY{n+nb}{len}\PY{p}{(}\PY{n}{weights}\PY{p}{)}\PY{p}{,} \PY{l+m+mi}{1}\PY{p}{,} \PY{n}{p}\PY{o}{=}\PY{n}{weights}\PY{p}{)}\PY{p}{[}\PY{l+m+mi}{0}\PY{p}{]}
        \PY{c+c1}{\PYZsh{} Use np.random.multivariate\PYZus{}normal to create data from this cluster}
        \PY{n}{x} \PY{o}{=} \PY{n}{np}\PY{o}{.}\PY{n}{random}\PY{o}{.}\PY{n}{multivariate\PYZus{}normal}\PY{p}{(}\PY{n}{means}\PY{p}{[}\PY{n}{k}\PY{p}{]}\PY{p}{,} \PY{n}{covariances}\PY{p}{[}\PY{n}{k}\PY{p}{]}\PY{p}{)}
        \PY{n}{data}\PY{o}{.}\PY{n}{append}\PY{p}{(}\PY{n}{x}\PY{p}{)}
    \PY{k}{return} \PY{n}{data}


\PY{c+c1}{\PYZsh{} Model parameters}
\PY{n}{data\PYZus{}means} \PY{o}{=} \PY{n}{np}\PY{o}{.}\PY{n}{array}\PY{p}{(}\PY{p}{[}
    \PY{p}{[}\PY{l+m+mi}{5}\PY{p}{,} \PY{l+m+mi}{0}\PY{p}{]}\PY{p}{,}  \PY{c+c1}{\PYZsh{} mean of cluster 1}
    \PY{p}{[}\PY{l+m+mi}{1}\PY{p}{,} \PY{l+m+mi}{1}\PY{p}{]}\PY{p}{,}  \PY{c+c1}{\PYZsh{} mean of cluster 2}
    \PY{p}{[}\PY{l+m+mi}{0}\PY{p}{,} \PY{l+m+mi}{5}\PY{p}{]}  \PY{c+c1}{\PYZsh{} mean of cluster 3}
\PY{p}{]}\PY{p}{)}
\PY{n}{data\PYZus{}covariances} \PY{o}{=} \PY{n}{np}\PY{o}{.}\PY{n}{array}\PY{p}{(}\PY{p}{[}
    \PY{p}{[}\PY{p}{[}\PY{o}{.}\PY{l+m+mi}{5}\PY{p}{,} \PY{l+m+mf}{0.}\PY{p}{]}\PY{p}{,} \PY{p}{[}\PY{l+m+mi}{0}\PY{p}{,} \PY{o}{.}\PY{l+m+mi}{5}\PY{p}{]}\PY{p}{]}\PY{p}{,}  \PY{c+c1}{\PYZsh{} covariance of cluster 1}
    \PY{p}{[}\PY{p}{[}\PY{o}{.}\PY{l+m+mi}{92}\PY{p}{,} \PY{o}{.}\PY{l+m+mi}{38}\PY{p}{]}\PY{p}{,} \PY{p}{[}\PY{o}{.}\PY{l+m+mi}{38}\PY{p}{,} \PY{o}{.}\PY{l+m+mi}{91}\PY{p}{]}\PY{p}{]}\PY{p}{,}  \PY{c+c1}{\PYZsh{} covariance of cluster 2}
    \PY{p}{[}\PY{p}{[}\PY{o}{.}\PY{l+m+mi}{5}\PY{p}{,} \PY{l+m+mf}{0.}\PY{p}{]}\PY{p}{,} \PY{p}{[}\PY{l+m+mi}{0}\PY{p}{,} \PY{o}{.}\PY{l+m+mi}{5}\PY{p}{]}\PY{p}{]}  \PY{c+c1}{\PYZsh{} covariance of cluster 3}
\PY{p}{]}\PY{p}{)}
\PY{n}{data\PYZus{}weights} \PY{o}{=} \PY{n}{np}\PY{o}{.}\PY{n}{array}\PY{p}{(}\PY{p}{[}\PY{l+m+mi}{1} \PY{o}{/} \PY{l+m+mf}{4.}\PY{p}{,} \PY{l+m+mi}{1} \PY{o}{/} \PY{l+m+mf}{2.}\PY{p}{,} \PY{l+m+mi}{1} \PY{o}{/} \PY{l+m+mf}{4.}\PY{p}{]}\PY{p}{)}  \PY{c+c1}{\PYZsh{} weights of each cluster}

\PY{n}{np}\PY{o}{.}\PY{n}{random}\PY{o}{.}\PY{n}{seed}\PY{p}{(}\PY{l+m+mi}{4}\PY{p}{)}
\PY{n}{data} \PY{o}{=} \PY{n}{np}\PY{o}{.}\PY{n}{array}\PY{p}{(}\PY{n}{generate\PYZus{}MoG\PYZus{}data}\PY{p}{(}\PY{l+m+mi}{100}\PY{p}{,} \PY{n}{data\PYZus{}means}\PY{p}{,} \PY{n}{data\PYZus{}covariances}\PY{p}{,} \PY{n}{data\PYZus{}weights}\PY{p}{)}\PY{p}{)}
\end{Verbatim}
\end{tcolorbox}

    Now plot the data you created above. The plot should be a scatterplot
with 100 points that appear to roughly fall into three clusters.

    \begin{tcolorbox}[breakable, size=fbox, boxrule=1pt, pad at break*=1mm,colback=cellbackground, colframe=cellborder]
\prompt{In}{incolor}{67}{\boxspacing}
\begin{Verbatim}[commandchars=\\\{\}]
\PY{n}{plt}\PY{o}{.}\PY{n}{figure}\PY{p}{(}\PY{p}{)}
\PY{n}{d} \PY{o}{=} \PY{n}{np}\PY{o}{.}\PY{n}{vstack}\PY{p}{(}\PY{n}{data}\PY{p}{)}
\PY{n}{plt}\PY{o}{.}\PY{n}{plot}\PY{p}{(}\PY{n}{d}\PY{p}{[}\PY{p}{:}\PY{p}{,} \PY{l+m+mi}{0}\PY{p}{]}\PY{p}{,} \PY{n}{d}\PY{p}{[}\PY{p}{:}\PY{p}{,} \PY{l+m+mi}{1}\PY{p}{]}\PY{p}{,} \PY{l+s+s1}{\PYZsq{}}\PY{l+s+s1}{ko}\PY{l+s+s1}{\PYZsq{}}\PY{p}{)}
\PY{n}{plt}\PY{o}{.}\PY{n}{rcParams}\PY{o}{.}\PY{n}{update}\PY{p}{(}\PY{p}{\PYZob{}}\PY{l+s+s1}{\PYZsq{}}\PY{l+s+s1}{font.size}\PY{l+s+s1}{\PYZsq{}}\PY{p}{:} \PY{l+m+mi}{16}\PY{p}{\PYZcb{}}\PY{p}{)}
\PY{n}{plt}\PY{o}{.}\PY{n}{tight\PYZus{}layout}\PY{p}{(}\PY{p}{)}
\PY{n}{plt}\PY{o}{.}\PY{n}{show}\PY{p}{(}\PY{p}{)}
\end{Verbatim}
\end{tcolorbox}

    \begin{center}
    \adjustimage{max size={0.9\linewidth}{0.9\paperheight}}{HW4-Coding_files/HW4-Coding_8_0.png}
    \end{center}
    { \hspace*{\fill} \\}
    
    Now we'll fit a mixture of Gaussians to this data using our
implementation of the EM algorithm. As with k-means, it is important to
ask how we obtain an initial configuration of mixing weights and
component parameters. In this simple case, we'll take three random
points to be the initial cluster means, use the empirical covariance of
the data to be the initial covariance in each cluster (a clear
overestimate), and set the initial mixing weights to be uniform across
clusters.

    \begin{tcolorbox}[breakable, size=fbox, boxrule=1pt, pad at break*=1mm,colback=cellbackground, colframe=cellborder]
\prompt{In}{incolor}{68}{\boxspacing}
\begin{Verbatim}[commandchars=\\\{\}]
\PY{c+c1}{\PYZsh{} Initialization of parameters}
\PY{n}{np}\PY{o}{.}\PY{n}{random}\PY{o}{.}\PY{n}{seed}\PY{p}{(}\PY{l+m+mi}{4}\PY{p}{)}
\PY{n}{chosen} \PY{o}{=} \PY{n}{np}\PY{o}{.}\PY{n}{random}\PY{o}{.}\PY{n}{choice}\PY{p}{(}\PY{n+nb}{len}\PY{p}{(}\PY{n}{data}\PY{p}{)}\PY{p}{,} \PY{l+m+mi}{3}\PY{p}{,} \PY{n}{replace}\PY{o}{=}\PY{k+kc}{False}\PY{p}{)}
\PY{n}{initial\PYZus{}means} \PY{o}{=} \PY{n}{np}\PY{o}{.}\PY{n}{array}\PY{p}{(}\PY{p}{[}\PY{n}{data}\PY{p}{[}\PY{n}{x}\PY{p}{]} \PY{k}{for} \PY{n}{x} \PY{o+ow}{in} \PY{n}{chosen}\PY{p}{]}\PY{p}{)}
\PY{n}{initial\PYZus{}covariances} \PY{o}{=} \PY{n}{np}\PY{o}{.}\PY{n}{array}\PY{p}{(}\PY{p}{[}\PY{n}{np}\PY{o}{.}\PY{n}{cov}\PY{p}{(}\PY{n}{data}\PY{p}{,} \PY{n}{rowvar}\PY{o}{=}\PY{k+kc}{False}\PY{p}{)}\PY{p}{]} \PY{o}{*} \PY{l+m+mi}{3}\PY{p}{)}
\PY{n}{initial\PYZus{}weights} \PY{o}{=} \PY{n}{np}\PY{o}{.}\PY{n}{array}\PY{p}{(}\PY{p}{[}\PY{l+m+mi}{1} \PY{o}{/} \PY{l+m+mf}{3.}\PY{p}{,} \PY{l+m+mi}{1} \PY{o}{/} \PY{l+m+mf}{3.}\PY{p}{,} \PY{l+m+mi}{1} \PY{o}{/} \PY{l+m+mf}{3.}\PY{p}{]}\PY{p}{)}
\end{Verbatim}
\end{tcolorbox}

    We will use the following plot\_contours() function to visualize the
Gaussian components over the data at three different points in the
algorithm's execution:

\begin{enumerate}
\def\labelenumi{\arabic{enumi}.}
\tightlist
\item
  At initialization (using initial\_means, initial\_covariances, and
  initial\_weights)
\item
  After running the algorithm to completion
\item
  After 20 iterations
\end{enumerate}

    \begin{tcolorbox}[breakable, size=fbox, boxrule=1pt, pad at break*=1mm,colback=cellbackground, colframe=cellborder]
\prompt{In}{incolor}{69}{\boxspacing}
\begin{Verbatim}[commandchars=\\\{\}]
\PY{k}{def} \PY{n+nf}{bivariate\PYZus{}normal}\PY{p}{(}\PY{n}{X}\PY{p}{,} \PY{n}{Y}\PY{p}{,} \PY{n}{sigmax}\PY{o}{=}\PY{l+m+mf}{1.0}\PY{p}{,} \PY{n}{sigmay}\PY{o}{=}\PY{l+m+mf}{1.0}\PY{p}{,}
                     \PY{n}{mux}\PY{o}{=}\PY{l+m+mf}{0.0}\PY{p}{,} \PY{n}{muy}\PY{o}{=}\PY{l+m+mf}{0.0}\PY{p}{,} \PY{n}{sigmaxy}\PY{o}{=}\PY{l+m+mf}{0.0}\PY{p}{)}\PY{p}{:}
    \PY{n}{Xmu} \PY{o}{=} \PY{n}{X} \PY{o}{\PYZhy{}} \PY{n}{mux}
    \PY{n}{Ymu} \PY{o}{=} \PY{n}{Y} \PY{o}{\PYZhy{}} \PY{n}{muy}
    \PY{n}{rho} \PY{o}{=} \PY{n}{sigmaxy} \PY{o}{/} \PY{p}{(}\PY{n}{sigmax} \PY{o}{*} \PY{n}{sigmay}\PY{p}{)}
    \PY{n}{z} \PY{o}{=} \PY{n}{Xmu} \PY{o}{*}\PY{o}{*} \PY{l+m+mi}{2} \PY{o}{/} \PY{n}{sigmax} \PY{o}{*}\PY{o}{*} \PY{l+m+mi}{2} \PY{o}{+} \PY{n}{Ymu} \PY{o}{*}\PY{o}{*} \PY{l+m+mi}{2} \PY{o}{/} \PY{n}{sigmay} \PY{o}{*}\PY{o}{*} \PY{l+m+mi}{2} \PY{o}{\PYZhy{}} \PY{l+m+mi}{2} \PY{o}{*} \PY{n}{rho} \PY{o}{*} \PY{n}{Xmu} \PY{o}{*} \PY{n}{Ymu} \PY{o}{/} \PY{p}{(}\PY{n}{sigmax} \PY{o}{*} \PY{n}{sigmay}\PY{p}{)}
    \PY{n}{denom} \PY{o}{=} \PY{l+m+mi}{2} \PY{o}{*} \PY{n}{np}\PY{o}{.}\PY{n}{pi} \PY{o}{*} \PY{n}{sigmax} \PY{o}{*} \PY{n}{sigmay} \PY{o}{*} \PY{n}{np}\PY{o}{.}\PY{n}{sqrt}\PY{p}{(}\PY{l+m+mi}{1} \PY{o}{\PYZhy{}} \PY{n}{rho} \PY{o}{*}\PY{o}{*} \PY{l+m+mi}{2}\PY{p}{)}
    \PY{k}{return} \PY{n}{np}\PY{o}{.}\PY{n}{exp}\PY{p}{(}\PY{o}{\PYZhy{}}\PY{n}{z} \PY{o}{/} \PY{p}{(}\PY{l+m+mi}{2} \PY{o}{*} \PY{p}{(}\PY{l+m+mi}{1} \PY{o}{\PYZhy{}} \PY{n}{rho} \PY{o}{*}\PY{o}{*} \PY{l+m+mi}{2}\PY{p}{)}\PY{p}{)}\PY{p}{)} \PY{o}{/} \PY{n}{denom}


\PY{k}{def} \PY{n+nf}{plot\PYZus{}contours}\PY{p}{(}\PY{n}{data}\PY{p}{,} \PY{n}{means}\PY{p}{,} \PY{n}{covs}\PY{p}{,} \PY{n}{title}\PY{p}{)}\PY{p}{:}
    \PY{n}{plt}\PY{o}{.}\PY{n}{figure}\PY{p}{(}\PY{p}{)}
    \PY{n}{plt}\PY{o}{.}\PY{n}{plot}\PY{p}{(}\PY{p}{[}\PY{n}{x}\PY{p}{[}\PY{l+m+mi}{0}\PY{p}{]} \PY{k}{for} \PY{n}{x} \PY{o+ow}{in} \PY{n}{data}\PY{p}{]}\PY{p}{,} \PY{p}{[}\PY{n}{y}\PY{p}{[}\PY{l+m+mi}{1}\PY{p}{]} \PY{k}{for} \PY{n}{y} \PY{o+ow}{in} \PY{n}{data}\PY{p}{]}\PY{p}{,} \PY{l+s+s1}{\PYZsq{}}\PY{l+s+s1}{ko}\PY{l+s+s1}{\PYZsq{}}\PY{p}{)}  \PY{c+c1}{\PYZsh{} data}

    \PY{n}{delta} \PY{o}{=} \PY{l+m+mf}{0.025}
    \PY{n}{k} \PY{o}{=} \PY{n+nb}{len}\PY{p}{(}\PY{n}{means}\PY{p}{)}
    \PY{n}{x} \PY{o}{=} \PY{n}{np}\PY{o}{.}\PY{n}{arange}\PY{p}{(}\PY{o}{\PYZhy{}}\PY{l+m+mf}{2.0}\PY{p}{,} \PY{l+m+mf}{7.0}\PY{p}{,} \PY{n}{delta}\PY{p}{)}
    \PY{n}{y} \PY{o}{=} \PY{n}{np}\PY{o}{.}\PY{n}{arange}\PY{p}{(}\PY{o}{\PYZhy{}}\PY{l+m+mf}{2.0}\PY{p}{,} \PY{l+m+mf}{7.0}\PY{p}{,} \PY{n}{delta}\PY{p}{)}
    \PY{n}{X}\PY{p}{,} \PY{n}{Y} \PY{o}{=} \PY{n}{np}\PY{o}{.}\PY{n}{meshgrid}\PY{p}{(}\PY{n}{x}\PY{p}{,} \PY{n}{y}\PY{p}{)}
    \PY{n}{col} \PY{o}{=} \PY{p}{[}\PY{l+s+s1}{\PYZsq{}}\PY{l+s+s1}{green}\PY{l+s+s1}{\PYZsq{}}\PY{p}{,} \PY{l+s+s1}{\PYZsq{}}\PY{l+s+s1}{red}\PY{l+s+s1}{\PYZsq{}}\PY{p}{,} \PY{l+s+s1}{\PYZsq{}}\PY{l+s+s1}{indigo}\PY{l+s+s1}{\PYZsq{}}\PY{p}{]}
    \PY{k}{for} \PY{n}{i} \PY{o+ow}{in} \PY{n+nb}{range}\PY{p}{(}\PY{n}{k}\PY{p}{)}\PY{p}{:}
        \PY{n}{mean} \PY{o}{=} \PY{n}{means}\PY{p}{[}\PY{n}{i}\PY{p}{]}
        \PY{n}{cov} \PY{o}{=} \PY{n}{covs}\PY{p}{[}\PY{n}{i}\PY{p}{]}
        \PY{n}{sigmax} \PY{o}{=} \PY{n}{np}\PY{o}{.}\PY{n}{sqrt}\PY{p}{(}\PY{n}{cov}\PY{p}{[}\PY{l+m+mi}{0}\PY{p}{]}\PY{p}{[}\PY{l+m+mi}{0}\PY{p}{]}\PY{p}{)}
        \PY{n}{sigmay} \PY{o}{=} \PY{n}{np}\PY{o}{.}\PY{n}{sqrt}\PY{p}{(}\PY{n}{cov}\PY{p}{[}\PY{l+m+mi}{1}\PY{p}{]}\PY{p}{[}\PY{l+m+mi}{1}\PY{p}{]}\PY{p}{)}
        \PY{n}{sigmaxy} \PY{o}{=} \PY{n}{cov}\PY{p}{[}\PY{l+m+mi}{0}\PY{p}{]}\PY{p}{[}\PY{l+m+mi}{1}\PY{p}{]} \PY{o}{/} \PY{p}{(}\PY{n}{sigmax} \PY{o}{*} \PY{n}{sigmay}\PY{p}{)}
        \PY{n}{Z} \PY{o}{=} \PY{n}{bivariate\PYZus{}normal}\PY{p}{(}\PY{n}{X}\PY{p}{,} \PY{n}{Y}\PY{p}{,} \PY{n}{sigmax}\PY{p}{,} \PY{n}{sigmay}\PY{p}{,} \PY{n}{mean}\PY{p}{[}\PY{l+m+mi}{0}\PY{p}{]}\PY{p}{,} \PY{n}{mean}\PY{p}{[}\PY{l+m+mi}{1}\PY{p}{]}\PY{p}{,} \PY{n}{sigmaxy}\PY{p}{)}
        \PY{n}{plt}\PY{o}{.}\PY{n}{contour}\PY{p}{(}\PY{n}{X}\PY{p}{,} \PY{n}{Y}\PY{p}{,} \PY{n}{Z}\PY{p}{,} \PY{n}{colors}\PY{o}{=}\PY{n}{col}\PY{p}{[}\PY{n}{i}\PY{p}{]}\PY{p}{)}
        \PY{n}{plt}\PY{o}{.}\PY{n}{title}\PY{p}{(}\PY{n}{title}\PY{p}{)}
    \PY{n}{plt}\PY{o}{.}\PY{n}{rcParams}\PY{o}{.}\PY{n}{update}\PY{p}{(}\PY{p}{\PYZob{}}\PY{l+s+s1}{\PYZsq{}}\PY{l+s+s1}{font.size}\PY{l+s+s1}{\PYZsq{}}\PY{p}{:} \PY{l+m+mi}{16}\PY{p}{\PYZcb{}}\PY{p}{)}
    \PY{n}{plt}\PY{o}{.}\PY{n}{tight\PYZus{}layout}\PY{p}{(}\PY{p}{)}
    \PY{n}{plt}\PY{o}{.}\PY{n}{show}\PY{p}{(}\PY{p}{)}
\end{Verbatim}
\end{tcolorbox}

    \begin{tcolorbox}[breakable, size=fbox, boxrule=1pt, pad at break*=1mm,colback=cellbackground, colframe=cellborder]
\prompt{In}{incolor}{70}{\boxspacing}
\begin{Verbatim}[commandchars=\\\{\}]
\PY{c+c1}{\PYZsh{} Parameters after initialization}
\PY{n}{plot\PYZus{}contours}\PY{p}{(}\PY{n}{data}\PY{p}{,} \PY{n}{initial\PYZus{}means}\PY{p}{,} \PY{n}{initial\PYZus{}covariances}\PY{p}{,} \PY{l+s+s1}{\PYZsq{}}\PY{l+s+s1}{Initial clusters}\PY{l+s+s1}{\PYZsq{}}\PY{p}{)}
\end{Verbatim}
\end{tcolorbox}

    \begin{center}
    \adjustimage{max size={0.9\linewidth}{0.9\paperheight}}{HW4-Coding_files/HW4-Coding_13_0.png}
    \end{center}
    { \hspace*{\fill} \\}
    
    Now run the EM algorithm and plot contours afterwards.

    \begin{tcolorbox}[breakable, size=fbox, boxrule=1pt, pad at break*=1mm,colback=cellbackground, colframe=cellborder]
\prompt{In}{incolor}{71}{\boxspacing}
\begin{Verbatim}[commandchars=\\\{\}]
\PY{c+c1}{\PYZsh{} Run EM}
\PY{n}{my\PYZus{}means} \PY{o}{=} \PY{n}{initial\PYZus{}means}
\PY{n}{my\PYZus{}weights} \PY{o}{=} \PY{n}{initial\PYZus{}weights}
\PY{n}{my\PYZus{}covariances} \PY{o}{=} \PY{n}{initial\PYZus{}covariances}
\PY{n}{iters} \PY{o}{=} \PY{l+m+mi}{20}
\PY{k}{for} \PY{n+nb}{iter} \PY{o+ow}{in} \PY{n+nb}{range}\PY{p}{(}\PY{n}{iters}\PY{p}{)}\PY{p}{:}
    \PY{c+c1}{\PYZsh{} Here we run 20 iterations. You can use a more strict convergence criterion,}
    \PY{c+c1}{\PYZsh{} such as that the increment of the log\PYZhy{}likelihood function is less than 1e\PYZhy{}3}
    \PY{n}{prob\PYZus{}matrix} \PY{o}{=} \PY{n}{probabilities}\PY{p}{(}\PY{n}{data}\PY{p}{,} \PY{n}{my\PYZus{}weights}\PY{p}{,} \PY{n}{my\PYZus{}means}\PY{p}{,} \PY{n}{my\PYZus{}covariances}\PY{p}{)}
    \PY{p}{[}\PY{n}{my\PYZus{}weights}\PY{p}{,} \PY{n}{my\PYZus{}means}\PY{p}{,} \PY{n}{my\PYZus{}covariances}\PY{p}{]} \PY{o}{=} \PY{n}{updates}\PY{p}{(}\PY{n}{data}\PY{p}{,} \PY{n}{prob\PYZus{}matrix}\PY{p}{,} \PY{n}{my\PYZus{}weights}\PY{p}{,} \PY{n}{my\PYZus{}means}\PY{p}{,} \PY{n}{my\PYZus{}covariances}\PY{p}{)}
\PY{n+nb}{print}\PY{p}{(}\PY{l+s+s2}{\PYZdq{}}\PY{l+s+s2}{my\PYZus{}weights =}\PY{l+s+se}{\PYZbs{}n}\PY{l+s+si}{\PYZob{}\PYZcb{}}\PY{l+s+s2}{,}\PY{l+s+se}{\PYZbs{}n}\PY{l+s+se}{\PYZbs{}n}\PY{l+s+s2}{my\PYZus{}means =}\PY{l+s+se}{\PYZbs{}n}\PY{l+s+si}{\PYZob{}\PYZcb{}}\PY{l+s+s2}{,}\PY{l+s+se}{\PYZbs{}n}\PY{l+s+se}{\PYZbs{}n}\PY{l+s+s2}{my\PYZus{}covariances =}\PY{l+s+se}{\PYZbs{}n}\PY{l+s+si}{\PYZob{}\PYZcb{}}\PY{l+s+s2}{\PYZdq{}}\PY{o}{.}\PY{n}{format}\PY{p}{(}\PY{n}{my\PYZus{}weights}\PY{p}{,} \PY{n}{my\PYZus{}means}\PY{p}{,} \PY{n}{my\PYZus{}covariances}\PY{p}{)}\PY{p}{)}
\end{Verbatim}
\end{tcolorbox}

    \begin{Verbatim}[commandchars=\\\{\}]
my\_weights =
[0.30100743 0.17993525 0.51905733],

my\_means =
[[0.02238127 4.94603136]
 [4.94240545 0.31364134]
 [1.08184868 0.73761235]],

my\_covariances =
[[[ 0.29398303  0.04872587]
  [ 0.04872587  0.35540776]]

 [[ 0.35562857 -0.01493194]
  [-0.01493194  0.66694074]]

 [[ 0.67155363  0.3308503 ]
  [ 0.3308503   0.90125037]]]
    \end{Verbatim}

    \begin{tcolorbox}[breakable, size=fbox, boxrule=1pt, pad at break*=1mm,colback=cellbackground, colframe=cellborder]
\prompt{In}{incolor}{72}{\boxspacing}
\begin{Verbatim}[commandchars=\\\{\}]
\PY{c+c1}{\PYZsh{} Parameters after running EM to convergence}
\PY{n}{plot\PYZus{}contours}\PY{p}{(}\PY{n}{data}\PY{p}{,} \PY{n}{my\PYZus{}means}\PY{p}{,} \PY{n}{my\PYZus{}covariances}\PY{p}{,} \PY{l+s+s1}{\PYZsq{}}\PY{l+s+s1}{Final clusters}\PY{l+s+s1}{\PYZsq{}}\PY{p}{)}
\end{Verbatim}
\end{tcolorbox}

    \begin{center}
    \adjustimage{max size={0.9\linewidth}{0.9\paperheight}}{HW4-Coding_files/HW4-Coding_16_0.png}
    \end{center}
    { \hspace*{\fill} \\}
    
    Now call the GaussianMixture() method in sklearn package, and see its
performance.

    \begin{tcolorbox}[breakable, size=fbox, boxrule=1pt, pad at break*=1mm,colback=cellbackground, colframe=cellborder]
\prompt{In}{incolor}{73}{\boxspacing}
\begin{Verbatim}[commandchars=\\\{\}]
\PY{k+kn}{from} \PY{n+nn}{sklearn}\PY{n+nn}{.}\PY{n+nn}{mixture} \PY{k}{import} \PY{n}{GaussianMixture}

\PY{n}{gmm\PYZus{}sklearn} \PY{o}{=} \PY{n}{GaussianMixture}\PY{p}{(}\PY{n}{n\PYZus{}components}\PY{o}{=}\PY{l+m+mi}{3}\PY{p}{,} \PY{n}{covariance\PYZus{}type}\PY{o}{=}\PY{l+s+s1}{\PYZsq{}}\PY{l+s+s1}{full}\PY{l+s+s1}{\PYZsq{}}\PY{p}{,}
                              \PY{n}{max\PYZus{}iter}\PY{o}{=}\PY{l+m+mi}{20}\PY{p}{,} \PY{n}{weights\PYZus{}init}\PY{o}{=}\PY{n}{initial\PYZus{}weights}\PY{p}{,}
                              \PY{n}{means\PYZus{}init}\PY{o}{=}\PY{n}{initial\PYZus{}means}\PY{p}{)}\PY{o}{.}\PY{n}{fit}\PY{p}{(}\PY{n}{data}\PY{p}{)}
\PY{n+nb}{print}\PY{p}{(}\PY{l+s+s2}{\PYZdq{}}\PY{l+s+s2}{weights\PYZus{}sklearn =}\PY{l+s+se}{\PYZbs{}n}\PY{l+s+si}{\PYZob{}\PYZcb{}}\PY{l+s+s2}{,}\PY{l+s+se}{\PYZbs{}n}\PY{l+s+se}{\PYZbs{}n}\PY{l+s+s2}{means\PYZus{}sklearn =}\PY{l+s+se}{\PYZbs{}n}\PY{l+s+si}{\PYZob{}\PYZcb{}}\PY{l+s+s2}{,}\PY{l+s+se}{\PYZbs{}n}\PY{l+s+se}{\PYZbs{}n}\PY{l+s+s2}{covariance\PYZus{}sklearn =}\PY{l+s+se}{\PYZbs{}n}\PY{l+s+si}{\PYZob{}\PYZcb{}}\PY{l+s+s2}{\PYZdq{}}\PY{o}{.}\PY{n}{format}\PY{p}{(}\PY{n}{gmm\PYZus{}sklearn}\PY{o}{.}\PY{n}{weights\PYZus{}}\PY{p}{,}
                                                                                          \PY{n}{gmm\PYZus{}sklearn}\PY{o}{.}\PY{n}{means\PYZus{}}\PY{p}{,}
                                                                                          \PY{n}{gmm\PYZus{}sklearn}\PY{o}{.}\PY{n}{covariances\PYZus{}}\PY{p}{)}\PY{p}{)}
\end{Verbatim}
\end{tcolorbox}

    \begin{Verbatim}[commandchars=\\\{\}]
weights\_sklearn =
[0.30089166 0.17993589 0.51917245],

means\_sklearn =
[[0.02199279 4.94669312]
 [4.9424009  0.31364545]
 [1.08183436 0.73816636]],

covariance\_sklearn =
[[[ 0.29370389  0.04941099]
  [ 0.04941099  0.3543986 ]]

 [[ 0.35563482 -0.01493779]
  [-0.01493779  0.66694506]]

 [[ 0.67139816  0.33074842]
  [ 0.33074842  0.90243587]]]
    \end{Verbatim}

    \begin{tcolorbox}[breakable, size=fbox, boxrule=1pt, pad at break*=1mm,colback=cellbackground, colframe=cellborder]
\prompt{In}{incolor}{74}{\boxspacing}
\begin{Verbatim}[commandchars=\\\{\}]
\PY{c+c1}{\PYZsh{} Parameters after running EM to convergence}
\PY{n}{plot\PYZus{}contours}\PY{p}{(}\PY{n}{data}\PY{p}{,} \PY{n}{gmm\PYZus{}sklearn}\PY{o}{.}\PY{n}{means\PYZus{}}\PY{p}{,} \PY{n}{gmm\PYZus{}sklearn}\PY{o}{.}\PY{n}{covariances\PYZus{}}\PY{p}{,} \PY{l+s+s1}{\PYZsq{}}\PY{l+s+s1}{Clusters by sklearn}\PY{l+s+s1}{\PYZsq{}}\PY{p}{)}
\end{Verbatim}
\end{tcolorbox}

    \begin{center}
    \adjustimage{max size={0.9\linewidth}{0.9\paperheight}}{HW4-Coding_files/HW4-Coding_19_0.png}
    \end{center}
    { \hspace*{\fill} \\}
    
    Now see the differences of parameters of your algorithm and that of
sklearn. You should see the precision is at least \textbf{1e-3}.

    \begin{tcolorbox}[breakable, size=fbox, boxrule=1pt, pad at break*=1mm,colback=cellbackground, colframe=cellborder]
\prompt{In}{incolor}{75}{\boxspacing}
\begin{Verbatim}[commandchars=\\\{\}]
\PY{n+nb}{print}\PY{p}{(}\PY{l+s+s2}{\PYZdq{}}\PY{l+s+s2}{After 20 iterations, differences between our algorithm and sklearn:}\PY{l+s+se}{\PYZbs{}n}\PY{l+s+s2}{\PYZdq{}}\PY{p}{)}
\PY{n+nb}{print}\PY{p}{(}\PY{l+s+s2}{\PYZdq{}}\PY{l+s+s2}{delta\PYZus{}weights =}\PY{l+s+se}{\PYZbs{}n}\PY{l+s+si}{\PYZob{}\PYZcb{}}\PY{l+s+s2}{,}\PY{l+s+se}{\PYZbs{}n}\PY{l+s+se}{\PYZbs{}n}\PY{l+s+s2}{delta\PYZus{}means =}\PY{l+s+se}{\PYZbs{}n}\PY{l+s+si}{\PYZob{}\PYZcb{}}\PY{l+s+s2}{,}\PY{l+s+se}{\PYZbs{}n}\PY{l+s+se}{\PYZbs{}n}\PY{l+s+s2}{delta\PYZus{}covariances=}\PY{l+s+se}{\PYZbs{}n}\PY{l+s+si}{\PYZob{}\PYZcb{}}\PY{l+s+s2}{\PYZdq{}}\PY{o}{.}\PY{n}{format}\PY{p}{(}\PY{n}{my\PYZus{}weights} \PY{o}{\PYZhy{}} \PY{n}{gmm\PYZus{}sklearn}\PY{o}{.}\PY{n}{weights\PYZus{}}\PY{p}{,}
                                                                                    \PY{n}{my\PYZus{}means} \PY{o}{\PYZhy{}} \PY{n}{gmm\PYZus{}sklearn}\PY{o}{.}\PY{n}{means\PYZus{}}\PY{p}{,}
                                                                                    \PY{n}{my\PYZus{}covariances} \PY{o}{\PYZhy{}} \PY{n}{gmm\PYZus{}sklearn}\PY{o}{.}\PY{n}{covariances\PYZus{}}\PY{p}{)}\PY{p}{)}
\end{Verbatim}
\end{tcolorbox}

    \begin{Verbatim}[commandchars=\\\{\}]
After 20 iterations, differences between our algorithm and sklearn:

delta\_weights =
[ 1.15770227e-04 -6.45837833e-07 -1.15124389e-04],

delta\_means =
[[ 3.88483410e-04 -6.61764743e-04]
 [ 4.55441691e-06 -4.11036858e-06]
 [ 1.43250214e-05 -5.54005410e-04]],

delta\_covariances=
[[[ 2.79146961e-04 -6.85120337e-04]
  [-6.85120337e-04  1.00916157e-03]]

 [[-6.25209629e-06  5.85161700e-06]
  [ 5.85161700e-06 -4.31164803e-06]]

 [[ 1.55477197e-04  1.01874776e-04]
  [ 1.01874776e-04 -1.18549656e-03]]]
    \end{Verbatim}


    % Add a bibliography block to the postdoc
    
    
    
\end{document}
